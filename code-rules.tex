\documentclass[10pt,letterpaper]{article}
\input{plos-settings}

\begin{document}
\vspace*{0.2in}

\begin{flushleft}
{\Large
\textbf\newline{10 quick tips for making your code last beyond your current job}
}
\newline
\\
Clare Dillon\textsuperscript{2},
Priyanka Ojha\textsuperscript{3},
Ian McInerney\textsuperscript{4},
Georg Link\textsuperscript{5},
Mala Kumar\textsuperscript{6},
Eman Abdullah AlOmar\textsuperscript{9},
Mohamed Wiem Mkaouer\textsuperscript{10},
David Lippert\textsuperscript{11},
Daniel R.\ McCloy\textsuperscript{12},
Bill Branan\textsuperscript{13},
David Pérez-Suárez\textsuperscript{14},
Sam Cunliffe\textsuperscript{15},
Chang Liao\textsuperscript{16},
Christoph Treude\textsuperscript{17},
Tobias Augspurger\textsuperscript{18},
Sylwester Arabas\textsuperscript{19},
Ethan P.\ White\textsuperscript{20},
Fang Liu\textsuperscript{21},
Geoffrey Lentner\textsuperscript{22},
David Eyers\textsuperscript{23},
Elena Findley-de Regt\textsuperscript{25},
Paola Corrales\textsuperscript{27},
Pieter Huybrechts\textsuperscript{28},
Phani Velicheti\textsuperscript{29},
Daniel Morillo-Cuadrado\textsuperscript{30},
Tommy Guy\textsuperscript{31},
Jan Ainali\textsuperscript{32}
\\
\bigskip
\textbf{2} CURIOSS, Lero, University of Galway, Ireland, \orcid{0009-0008-6205-0296}\\
\textbf{3} \orcid{0000-0002-6844-6493}\\
\textbf{4} \orcid{0000-0003-2616-9771}\\
\textbf{5} Bitergia, \orcid{0000-0001-6769-7867}\\
\textbf{6} unknown\\
\textbf{9} \orcid{0000-0003-1800-9268}\\
\textbf{10} \orcid{0000-0001-6010-7561}\\
\textbf{11} \orcid{0009-0003-6444-9595}\\
\textbf{12} \orcid{0000-0002-7572-3241}\\
\textbf{13} \orcid{0000-0002-4735-6624}\\
\textbf{14} Centre for Advance Research Computing, University College London, UK \orcid{0000-0003-0784-6909}\\
\textbf{15} Centre for Advance Research Computing, University College London, UK \orcid{0000-0003-0167-8641}\\
\textbf{16} \orcid{0000-0002-7348-8858}\\
\textbf{17} \orcid{0000-0002-6919-2149}\\
\textbf{18} Unknown\\
\textbf{19} \orcid{0000-0003-2361-0082}\\
\textbf{20} Department of Wildlife Ecology and Conservation, University of Florida, Gainesville, FL, USA \orcid{0000-0001-6728-7745}\\
\textbf{21} \orcid{0000-0002-3383-2191}\\
\textbf{22} \orcid{0000-0001-9314-0683}\\
\textbf{23} \orcid{0000-0002-7284-8006}\\
\textbf{25} Unknown\\
\textbf{27} \orcid{0000-0003-1923-9129}\\
\textbf{28} Research Institute for Nature and Forest (INBO), Brussels, Belgium, \orcid{0000-0002-6658-6062}\\
\textbf{29} \orcid{0009-0004-2580-3624}\\
\textbf{30} Universidad Nacional de Educación a Distancia (UNED), Spain, \orcid{0000-0003-3021-3878}\\
\textbf{31} \orcid{0009-0003-3652-5036}\\
\textbf{32} \orcid{0000-0001-8747-1670}\\

\end{flushleft}

\section*{Abstract}

FIXME

\section*{Author summary}

FIXME

\linenumbers

\section*{Introduction}

Academic life is often precarious,
particularly for those who work on research software,
an asset that is essential to modern research \cite{Pearson2025}
but often underfunded, undercited, and overlooked \cite{Carver2022}.
The situation has recently worsened dramatically due to changes in the political climate:
researchers in all disciplines face the loss of funding or jobs,
a smaller pool of incoming students \cite{Mallapaty2025},
and institutional policies that prioritize commercial returns.
Researchers in non-academic organizations
such as government labs, non-governmental organizations (NGOs), and non-profits are also impacted \cite{Woodward2025}
as national research strategies are upended
and associated public funding redirected or cut entirely \cite{Nature2025}.

In this environment it is essential that research software projects be \emph{resilient}.
Code should outlast situations the author changing workplaces or careers,
partnernships failing,
a lab closing down,
or tooling and digital infrastructure becoming unavailable.
We therefore present FIXME tips to ensure your software remains accessible and usable by others
even if you are no longer able to work on it.
You do not need to adopt all of them to make things better;
instead,
as in all emergencies,
you should do what you can,
where you are,
with what you have.

This guide focuses on software.
Other guides already exist for data \cite{Perkel2023},
and archives such as \href{https://zenodo.org/}{Zenodo},
\href{https://figshare.com/}{FigShare},
the \href{https://eotarchive.org/}{End Of Term Archive}
or the \href{http://archive.org/}{Internet Archive}
are now widely used to ensure long-term availability.
However,
we have not found any guide to making research software resilient
or to \emph{sunsetting} it (i.e., phasing it out gracefully).
Even if your job is not in danger,
these tips will help others access and use your code,
contribute to your project,
and to reproduce and cite your work.
As such,
our recommendations
are part of the global push for more equitable assessment of research contributions
by initiatives like \href{https://sfdora.org/}{DORA}, \href{https://coara.eu/}{coARA}, and \href{https://adore.software/}{ADORE.software}.

\section*{1. Consider your threat model}

When making plans, it's useful to know what you're planning \emph{for}.
Being explicit about \textbf{threat models} helps you prioritize,
build consensus with colleagues,
and check whether you have forgotten something important.
As with all disasters,
making those plans before you need them helps reduce the odds of them being needed,
as planning may help you identify risks you can eliminate.

\begin{enumerate}
\item
  \textbf{Individual threats} affect one or a few members of your team,
  such as a foreign student having their visa revoked without notice
  or a contributor taking extended leave.
  Especially for software under the maintenance of a single author,
  individual threats can also come from more mundane career or life changes.
  The most common way to prepare for this is to require everyone to document their work,
  but that rarely works in practice:
  \begin{enumerate}
  \item
    The hours spent writing those descriptions are hours \emph{not} spent doing research,
    so people will always short-change the former to focus on the latter.
  \item
    People invariably fail to write down the ``obvious'' parts of their work
    that are anything but obvious to the next person.
  \end{enumerate}
  Section~\ref{FIXME} explores alternatives,
  particularly ones that can be put in place on short notice.

\item
  \textbf{Leadership threats} are individual threats that affect the project's leader,
  such as the leader being doxxed or targeted personally in the media because of their work.
  Section~\ref{FIXME} discusses ways to plan for this.

\item
  \textbf{Institutional threats} affect large groups at once,
  such as your department being shut down
  or your entire field having funding cut.
  These events affect so many people at the same time
  that the rest of the community can't absorb them.
  Regional and national governments plans for disasters like these
  by having evacuation plans to get victims to safe(r) places
  and corollary plans for putting beds in high school gyms
  and flying in food and emergency medical personnel to help people when they arrive.
  At the time of writing,
  universities and researchers' professional societies have not started to do the equivalent.

\item
  \textbf{Global threats} affect everyone,
  not just researchers.
  For example,
  there is no technical or legal obstacle to the US government
  requiring American companies to charge for video conferencing calls
  involving participants outside the United States.
  Similar levies on email,
  file storage,
  and other online services would paradoxically have \emph{less} long-term impact on research
  than the targeted threats described above,
  as they would force national governments to find effective remedies quickly.

\end{enumerate}

\section*{2. Stay within the law}

The short version:
Ensuring your work remains usable is not worth putting yourself at legal risk.
Before following any of the tips below,
make sure you have a legal right to do so.

The long version:
Most institutions and journals now have policies for licensing code and/or releasing it publicly \cite{Katz2018,Ham2019}.
Funding agencies at various levels also often have policies
(e.g.,
EU\footnote{\url{https://commission.europa.eu/about/departments-and-executive-agencies/digital-services/open-source-software-strategy_en}}
and NASA\footnote{\url{https://science.nasa.gov/open-science/nasa-open-science-funding-opportunities}})
which may or may not align with those of institutions and journals,
and institutional policies may place restrictions on what you are allowed to say publicly about your situation.
Our first actionable tip is therefore to find out what those policies are,
e.g.,
whether you need explicitly permission to publish your software,
and if so,
from whom.

If you are unsure whether there is a policy or not,
ask your direct manager,
the person who pays you,
or offices like the research office,
the library,
or the tech transfer office.
If your institution has a dedicated
\href{https://sustainoss.org/academic-map/universities/index.html}{Open Source Program Office} (OSPO),
ask them
or connect to networks like \href{https://curioss.org}{CURIOSS}
or the \href{https://todogroup.org/}{TODO Group}
that advocate for OSPOs.
If you work with multiple institutions,
look at your contract and the bilateral agreements around products or deliverables.

Once you know whether your institution has a policy on licensing code,
get whatever sign-offs you need immediately.
Reach out to colleagues who can review your code if that is necessary for publishing it,
and review theirs if they ask.
If it appears that there is no formal sign-off process,
send an email to someone in authority
(e.g., the chair of your department or your grant officer)
saying explicitly that you believe this to be the case,
and copy that email to an account you will be able to access
after leaving your institution.

Do not assume that if you had permission before,
you have it now or will have it in the future.
Policies are changing rapidly,
and you may find yourself locked into one that no longer allows you to do what you want.
Also consider that the person you report to may be replaced with one who knows you less well or is less sympathetic,
so acting now may be easier than acting later.

Finally,
if you have your next job lined up,
ask about their policies
and make sure that your right to share your work is written into your contract.
Making your code comply with policy after the fact is riskier and more time-consuming than doing so early,
and sometimes not possible.

\section*{3. Choose an open license}

Making code publicly available does not ensure that other people can use it:
you maintain all copyright unless you explicitly include a license,
so they may still need explicit permission
or may simply be unsure whether they do or not.
The Open Source Initiative (OSI)
maintains a list of \href{https://opensource.org/licenses}{open source licenses} it has approved;
their implications are widely understood,
so choosing one of those will make your project more understandable to others.
Please do \emph{not} try to write your own license,
as it will have the opposite effect.

Broadly speaking,
the MIT, Apache 2.0, or BSD-3-Clause licenses place the fewest restrictions on re-use.
The GPLv3 and AGPL-3.0 licenses are \emph{copyleft} licenses
that require people to share their modifications to the software with the community.
These licenses can prevent companies from taking advantage of your work without giving anything in return,
but institutions discourage or disallow the use of copyleft licenses
out of fear that they will constrain commercialization.

More broadly,
\href{https://creativecommons.org/}{Creative Commons} licenses can be used for documentation or research reports,
while \href{https://ethicalsource.dev/}{ethical licenses}
or licenses based on the \href{https://blueoakcouncil.org/license/1.0.0}{Blue Oak Model License} may also be appropriate.
For guidance,
go to \href{http://choosealicense.com}{choosealicense.com}
or refer to \cite{Fogel2020,Furtonato2021};
the discussion in \cite{Gomes2022} about why people \emph{don't} share is also helpful.
Whatever license you choose,
place its text in a file called \texttt{LICENSE}, \texttt{LICENSE.txt}, or \texttt{LICENSE.md}
in the root directory of your project,
as this is where people and automated tools alike will look for it.

\section*{4. Save everything in multiple places}

Now that you're legally able to share your code, remember LOCKSS: Lots of Copies Keep Stuff Safe.
But copies are not enough:
the threats to your project are political as well as technological,
so you should ensure that loss of institutional support
(or worse, that institution turning on you)
does not mean loss of access.

\href{https://github.com/}{GitHub},
\href{https://gitlab.com}{GitLab},
and \href{https://bitbucket.org/}{BitBucket}
are social coding platforms centered around Git,
a widely-used open source version control tool.
However,
they are all commercial entities based in the same legal jurisdiction,
which means they are vulnerable to \emph{correlated threats}:
trouble with any of them may mean trouble with all of them.
\href{https://codeberg.org/}{Codeberg},
while currently much smaller,
is a non-profit based in a different jurisdiction;
consider mirroring your work there or using it as your primary host.

When you create projects on hosted services,
use a team account as the project owner rather than a personal account.
Doing this makes it easier to give other people permission to manage the project,
and as noted in \ref{FIXME},
a project with multiple owners from different institutions
is harder for any one institution to lock down.
For the same reason,
do not rely solely on logins or email addresses associated with your institution:
instead,
ask yourself if you will still have access to the project if you lose that identity,
and add an identity you personally control to the team that owns the project.
You can also add your \href{https://orcid.org/}{ORCID} to the project's metadata
so that the project links to a profile that you control
even when your contact address changes.

\begin{quote}
  \noindent
  \textbf{What's In a Project?}

  When deciding what to store where,
  remember that your version control repository only stores part of what makes up your project.
  For example,
  GitHub issues and wikis are displayed in the browser but live in GitHub's database;
  you can use the GitHub API to download their content,
  but (a) the result is intended for consumption by machines, not people,
  and (b) if you are doing this on short notice,
  the odds are high that other people are trying to do it at the same time.
  GitHub's servers may not be able to manage that load when you need them to,
  so use tools like \href{https://github.com/jlord/offline-issues}{offline-issues}
  to save hosted content as plain-text files on a regular basis.
\end{quote}

Version control isn't the only way to create and save copies of your work.
\href{https://www.softwareheritage.org/how-to-archive-reference-code/}{Software Heritage} archives software from multiple forges;
you can also snapshot the current state of repositories in a compressed archive file
(e.g., \texttt{.zip} or \texttt{.tar.gz})
and deposit those copies with the \href{https://osf.io/}{Open Science Foundation} (OSF),
\href{https://zenodo.org/}{Zenodo},
or \href{https://figshare.com/}{figshare}.
(Zenodo even integrates with GitHub so that tagging a release on GitHub
automatically triggers deposit of a new archive on Zenodo,
but again,
this automation is only useful as long as both ends are accessible.)
Institutional,
\href{https://amt.coretrustseal.org/certificates/}{national},
or \href{https://safeguar.de/}{international} data repositories also enhance longevity.

Storing copies on someone else's computer isn't the only option:
you can (and should) make copies of your projects on computers that you own.
Again,
this is a place where colleagues can help:
ask them to make copies on their computers in exchange for you making copies of theirs on yours.
Whatever you do,
document it \href{https://ropensci.org/blog/2022/03/22/safeguards-and-backups-for-github-organizations/}{like rOpenSci has}.

Similarly,
if you distribute your code as a package on \href{https://pypi.org/}{PyPI},
\href{https://anaconda.org/anaconda/conda}{Conda},
or \href{https://cran.r-project.org/}{CRAN},
that package can contain all of the source code.
Doing this also fosters community collaboration

\section*{5. Encourage community adoption}

Publishing your code is not the same as publicizing it.
the more people who know about and rely on your project,
the greater the odds that someone will help keep it alive,
whether by working on directly or supporting you to do so.
A simple way to help people get involved is to label entry-level issues
(e.g., to use the ``good first issue'' label on GitHub)
so that people can find a place to start \cite{SteinmacherXXXX}.

Announcing your project on mailing lists, forums, or social media,
and talking about it at conferences are necessary but not sufficient:
everyone else is doing this too,
so your project will almost certainly get lost in the noise.
\cite{Kuchner2011} is a good guide to what you can do beyond the obvious.
A short video showing how to use the software
or a slide deck that others can incorporate into their lectures
will increase uptake by lowering the cost of adoption.
Similarly,
a one-page website that opens with an elevator pitch
explaining who the software is for
and how it will make their lives better
is much more likely to lead to that crucial ``second glance''
than a list of publications.

Disseminating your code via tutorials at conferences is another effective strategy.
Tutorials are an opportunity for you,
as an author,
to describe your work in depth
and (more importantly) convince your audience to use it.
Networking through such events will enable you to build relationships with
the people most likely to care about your project,
and is the best way to ensure that you land well after you leave your current position.

\begin{quote}
  \noindent
  \textbf{One to Many}

  If you can, have at least one person outside your organization commit to your code.
  Such community contributions lead to joint ownership of intellectual property (IP),
  which makes it harder for any single institution to lock down your work.
  You can reinforce this by adding their name to the copyright statement in your license.
\end{quote}

Finally,
remember that it isn't all about \emph{your} project.
If you ask others to help maintain your code,
be prepared to give them something in return:
a letter of reference,
help testing or documenting their project,
or co-authorship credit on the project and associated publications.

\section*{6. Manage the project properly (if you have time)}

FIXME: consolidate all this material as ``do what we've been telling you to do all along''.

FIXME: do this for credit as well as to be able to keep using the software.
Code is increasingly cited by researchers \cite{Smith2016,Katz2021,Garijo2024}
and recognized as a research object in its own right.

Create a DOI for each release of your software
to ensure citability and retrievability even if the project repository becomes unavailable.
As noted in the previous tip,
Zenodo can do this automatically if your project is hosted on GitHub,
but you can do it manually if you are hosting your project elsewhere.

Add a \href{https://citation-file-format.github.io/}{Citation.CFF}
or \href{https://codemeta.github.io}{codemeta.json} file to your code \cite{Druskat2021}.
The format captures the metadata associated with the software in the repository,
which in turn helps to ensure that it is citable by making it easy for others to do so.
This will enable you to get credit down the road without much effort,
especially as some generators and tools
such as \href{https://citation-file-format.github.io/cff-initializer-javascript/}{cffinit},
\href{https://codemeta.github.io/codemeta-generator/}{codemeta generator},
and \href{https://github.com/citation-file-format/citation-file-format/blob/main/README.md\#tools-to-work-with-citationcff-files-wrench}{others} exist.

If you have the time,
consider having your code peer-reviewed and published.
There are now many places where you can do this,
including the \href{https://joss.theoj.org/}{Journal of Open Source Software},
the \href{https://openresearchsoftware.metajnl.com/}{Journal of Open Research Software},
the \href{http://www.jstatsoft.org}{Journal of Statistical Software},
the \href{https://journal.r-project.org/}{R Journa}l
and \href{https://ropensci.org/}{rOpenSci}.
Make sure to reference any code,
including your own,
in any papers that use it.

Another option is publishing your normal work,
but including your code with that work.
Submitting your software as a part of the supplementary material of a journal article that uses it
not only improves the reproducibility of your analyses
but also acts as a snapshot backup of your software on the publisher's website.

There is work underway to create types of persistent identifiers for research software,
such as \href{https://interoperable-europe.ec.europa.eu/collection/open-source-observatory-osor/news/swhid-intrinsic-identifier-software-artefacts}{Software Heritage's work on SWHIDs}
or \href{https://arxiv.org/abs/2501.10415}{this proposal},
but it is still nascent.

Your code is only as useful as the use others can make of it.
Document it as much as is necessary for others to be able to pick it up and use it as intended.
Don't go overboard:
if you're reading this,
you don't have time to document every method.

In particular,
add a \texttt{README.md} at the root of your repository
that explains the purpose, setup, and usage of your work.
If you can,
add a tutorial or an explanation for how to use the major functions of the code.
Adding a minimal example that shows how to use it makes it easier for others to get started.
If it is a component or a library,
try showing how to integrate it into existing systems.
Consider sharing a screen recording of you using your code.
Instructions on how to build,
launch, test, or deploy it are also valuable,
but a README is the first step.
There are guides on writing documentation that may help
\cite{Huybrechts2024,Littauer2025,Katz2025,Turing2025}.

If you can, make the link to documentation easy to find, or available at a canonical URL.
Bundling and archiving documentation together with the code would also help.

A contributor's guide that explains how to set up a development environment,
how to add tests,
and how to become part of the project's team can be extremely helpful.
Having one of these also leads to an increase in contributions \cite{Sholler2019}.
Your project is more than just your code:
it is your governance,
your issue tracker and changelogs,
and your shared values.
If you can, write these down.

If you haven't named your code yet,
consider naming it in an inclusive way
without hinting at direct and exclusive affiliation to a single institution.
This can help if the code needs to be relocated,
but also in herding developers from other institutions to contribute.

There are many other steps that are possible for documenting your code \cite{Lee2018}.
It is worth repeating that you do not need to go overboard.
Documentation is never finished,
and documentation is an art,
not a checklist to be filled out.

Code can only deliver its full value if it can be re-executed.
Reproducibility is an essential part of making your code last.
We touched on this above with the call to document your code.
Documenting the code includes describing dependencies in a machine-readable way,
e.g., Python's \href{https://packaging.python.org/en/latest/guides/writing-pyproject-toml/}{pyproject.toml} file,
an npm \href{https://docs.npmjs.com/cli/v10/configuring-npm/package-json?v=true}{package.json} file,
or equivalent manifests.
When referencing code in your docs or manifests,
ensure that a version label is included
so that future users have a way to identify the proper version.
Versioning also helps in discerning and identifying portions of the code contributed under different affiliations.

Making your code independently executable with consistent outcomes
is more than just documenting what it needs to run.
It involves abstracting the code,
thinking about how it is used,
and then refactoring it to ensure that it is easier for the next person.
It also involves considering all of the steps necessary for your code to run.
That may also mean adding instructions on how to make code run on other machines under different environments.
If you have access to other machines,
you can attempt to run your code on them
and document the process as you go.
Providing instructions on how to reproduce your results in the cloud
enables reviewers and followers of your work to skip time-consuming environment setup
and dive straight into running the code.
If you use Jupyter notebooks for your work \cite{Perkel2018}
and host them in public repositories,
enabling their execution in the cloud can be reduced to a single URL-click
with platforms such as \href{http://mybinder.org}{mybinder.org}
or analogous proprietary solutions such as Google Colab
or \href{https://github.com/features/codespaces}{Github Codespaces}.

If you can, version your runtime environment.
Use tools like Docker, Podman, renv, conda, or rix to make clear
what anyone else would need to set up
to make sure the code is being reproduced in the same computational environment.

Try and prevent avoidable hardware or software vendor lock-in
that might limit usability of code by another institution.
If your code or the process you use to work on it
depends upon access to platforms that you may not keep,
see if you can find alternatives.
For instance,
look into simulation solutions on
\href{https://www.softwarepreservationnetwork.org/emulation-as-a-service-infrastructure/}{EaaSI}).
Some proprietary software solutions do have open source alternatives,
and it is worth documenting if and how one may leverage them to run your code.
When developing code for specific hardware such GPUs,
it is worth exploring frameworks that enable the same code to execute---albeit
with reduced performance---on standard commodity hardware.
Finally,
work with common tools if you can.
They'll make adoption easier,
as they're generally more portable than obscure projects.

Data and code live within a shared context.
As previously mentioned, there are other resources for making your data available.
Data that is necessary to run or reproduce your code should be made publicly available when possible.
In order for accessible data to be useful,
you may need to do some work with your code to ensure that others can use it.

Document your datasets and include metadata describing the data.
Even if you need to exclude sensitive data,
you may be able to provide a safe synthetic dataset,
or include metadata describing the data so that someone else could reasonably do so.

If your code references specific datasets or formats,
try to link to them in your documentation.
If your code was used in any publications,
you can list those publications in your documentation, too.

Avoiding proprietary formats for your data will help make your code more relevant.
Use CSV, JSON or other types of machine-readable data.
PDFs lock in the data and are difficult to mine.
Other formats are preferable for most data.
Adjust your code accordingly.

If you can, document analysis pipelines using Jupyter Notebooks,
R Markdown,
or any standard documentation.
Provide Makefiles or other workflows to automate data processing.
Ultimately,
others may have to make new data to work with your code if yours is not used for a significant period of time.
Ensuring that those who follow you---including your future self---can do so
will help extend the reach and impact of your code.

\section*{7. Cut corners if you have to act quickly}

FIXME: paramedics don't do what doctors do,
so go through all of the points listed above and propose hurry-up alternatives.

\section*{8. Write a succession plan}

  One way to prepare is to have a designated successor
  (who can also stand in for you if you ever want to take a holiday).
  Another is to talk with peers about who will inherit what if your project is shut down.

All of this work is in the expectation that the code may be used later.
Even if it isn't,
you want to leave it in such a state that it could be.
A succession plan can be used to suss out how others can take over;
a sunsetting plan can be used as a framework for how to mothball the code permanently.
Writing either plan will help you know what you're going to do to ensure that the code,
data,
and project are bundled up and put away nicely.

This is also useful because this process can have a deliverable in itself:
a statement of how the project was used and what you accomplished with it.
Write up the story of the project.
Share it widely.
This will make it easier to wind your project down
while also recognizing that all projects eventually end.

Write up the goals you initially had for this project
and ask if they were served by the project and whether you succeeded.
Sometimes,
it may make sense to fold a project into another one,
or to ask someone else to take it over.
The process of writing up a sunsetting plan can involve realizing that
the project doesn't need to end,
but your continued involvement may not serve the project.
Think about whether someone else could be trained to use it,
or take up leadership for it,
or fund work on it.

For some projects,
it may be worth working with a fiscal sponsor such as
\href{https://numfocus.org/}{NumFOCUS},
\href{https://sfconservancy.org/}{Software Freedom Conservancy},
\href{https://www.eclipse.org/}{the Eclipse Foundation},
or \href{https://oscollective.org/}{Open Source Collective}.
There are \href{https://sustainoss.org/academic-map/organizations/index.html}{many foundations and fiscal hosts}.
Through fiscal hosting,
your project can take donations from its community of users to fund continued maintenance or other project costs.
Universities or governments are generally not ideal fiscal hosts for software projects,
due to high overhead costs and procurement processes.
Smaller hosts like Open Source Collective take a minimal overhead in comparison,
often an order of magnitude less.
Some fiscal hosts are more hands on,
and may help offload much of the work of keeping your code going.
A key part of using a fiscal host is that
they put the software in a neutral home where it is no longer owned by a research performing organization.
This is based on the sponsor owning the intellectual property.
This also lets people become owners of the project over time
without having to move to a single institution.

Before you go,
remember to empower at least one other person to act on behalf of the project.
If you use GitHub,
consider designating a
\href{https://docs.github.com/en/account-and-profile/setting-up-and-managing-your-personal-account-on-github/managing-access-to-your-personal-repositories/maintaining-ownership-continuity-of-your-personal-accounts-repositories}{successor account}
that can administer your repositories if you can no longer access or control them.
If you don't have a successor designated,
it still helps to write up a document explaining where the keys are,
and how to use them.

\section*{9. Talk about what you're doing}

All of this work is labor.
You'll learn that not all of these steps apply to you,
and you'll find other steps that could also be useful.
Further, you may not want to do any of this, and it may feel like pulling teeth.

Make that work meaningful.
Talk about how hard it is to archive working code.
Talk about how necessary the code you've written is.
Talk about how others should protect and harden their systems.

You can also use this process to find more contributors,
or get others to hold it for you,
or to get sign off.
This may help your project live longer and help others use it.

Use Mastodon to find other archives that will be able to host your code in the future.
These steps will help ensure your work remains accessible and usable,
regardless of your employment situation.

Your institution may be able to help you,
with advance notice.
If your institution has an OSPO,
ask them to help with this process.
If not,
the library or the research office may also have helpful guides.

\section*{10. Organize.}

FIXME:
There's no point sharing what you have learned from the current disaster
if you don't use that knowledge to prevent or mitigate the next one.
None of us can fix this on our own;
become active in your professional society and/or advocacy groups.

\section*{Conclusion}

If you learn things, bring them back to the authors of this paper.
The world is a scary place.
\cite{Tamburri2020}
  There are calls for digital sovereignty coming from governments and lobby groups
  that may also influence where code can and should be stored and shared,
  including national efforts in
  \href{https://www.reuters.com/world/europe/dutch-parliament-calls-end-reliance-us-software-2025-03-18/}{the Netherlands},
  \href{https://www.sovereign.tech/}{Germany},
  and \href{https://zenodo.org/records/15080979}{New Zealand},
  the \href{https://www.politico.eu/article/push-for-eurostack-as-eu-us-tech-tensions-grow/}{proposal for a EuroStack}.

\bibliography{code-rules}

\end{document}
